This chapter provides a high-level overview of requirements engineering, requirements tooling, ReqIF and the terminology.

% ===================================================================================
\section{Requirements Engineering \& Management}
\label{sec:requirements_engineering}
\index{requirements engineering}
% ===================================================================================

This book is concerned with \pror{} a tool for requirements engineering.

\begin{definition}[Requirements Engineering]
\index{requirements engineering}
``Requirements engineering (RE) refers to the process of formulating, documenting and maintaining software requirements'' (\href{http://en.wikipedia.org/wiki/Requirements_engineering}{Wikipedia}).

We'd argue that RE also includes \textit{system} requirements.  Further, requirements are typically unstructured natural language.  Of high interest these days is model-driven requirements engineering.
\end{definition}

But engineering the requirements is not enough: they need to be \textit{managed}.

\begin{definition}[Requirements Managemenet]
\index{requirements management}
``Requirements management is the process of documenting, analyzing, tracing, prioritizing and agreeing on requirements and then controlling change and communicating to relevant stakeholders. It is a continuous process throughout a project.''
\end{definition}

% ===================================================================================
\section{Tools}
\label{sec:re-tools}
\index{tools}
% ===================================================================================

There are many tools available for requirements engineering.  These include free or cheap ones, like Microsoft Word and Excel, Wikis and issue trackers.  There are expensive, professional ones available, like IBM Rational DOORS, PTC Integrity or Visure IRQA.  Lately, there are also web-based tools, like Polarion.

\pror{} falls into the category of free tools.  But compared to the ones mentioned, it contains important features from professional tools, including traceability and typed attributes.  Further, by taking advantage of the Eclipse ecosystem, the tool can be augmented by plug-ins for version support, model integration, and much more.

\begin{info}
  Professional support, commercial components and integration services are available from \href{http://formalmind.com}{Formal Mind} and other service providers.
\end{info}


% ===================================================================================
\section{Requirements Interchange Format (ReqIF)}
\label{sec:reqif}
\index{ReqIF}
\index{Requirements Interchange Format}
% ===================================================================================

ReqIF stands for Requirements Interchange Format.  It is an exchange format for requirements and a data model.  \pror{} is an editor that can directly view and modify ReqIF data.

ReqIF was created to support the exchange of requirements across organizations.  For instance, it allows a manufacturer to send requirements to suppliers.  The suppliers can then comment and review the requirements, or they can create a system specification that is linked to the requirements.

ReqIF is an (\href{http://www.omg.org/spec/ReqIF/}{official OMG standard}).

\begin{warning}
ReqIF uses its own terminology.  Section~\ref{sec:terminology} defines the ReqIF vocabulary and how it relates to the terms used in classical requirements engineering.
\end{warning}

% -----------------------------------------------------------------------------------
\subsection{ReqIF History}
\label{sec:history}
\index{history}
% -----------------------------------------------------------------------------------

For technical and organizational reasons, two companies in the manufacturing industry are rarely able to work on the same requirements repository and sometimes do not work with the same requirements authoring tools.  A generic, non-proprietary format for requirements information is required to cross the chasm and to satisfy the urgent industry need for exchanging requirement information between different companies without losing the advantage of requirements management at the organizations' borders.

The Requirements Interchange Format (ReqIF) described in this RFC defines such a tool-independent exchange format.  Requirement information is exchanged by transferring XML documents that comply to the ReqIF format.

In 2004, the HIS (Hersteller Initiative Software), a panel of Germany's automotive manufacturers (Daimler, VW, Porsche, Audi and BMW Group) developed the idea of creating the ``Requirements Interchange Format''.  In 2005, the first version of that format was presented at the REConf®, a conference about requirements engineering and management, in Munich.  In 2008, the HIS Steering Committee decided that the internationalization and maintenance of the Requirements Interchange Format should be proceeded with the ProSTEP iViP Association.  A project was set up and a team was built that includes members of the ProSTEP iViP Association, representatives of manufacturing companies (Audi, BMW  Group, Daimler, VW, Bosch and Continental), tool vendors (Atego, IBM, MKS) and development partners (HOOD GmbH, PROSTEP AG).

\begin{info}
Further reading: \href{http://formalmind.com/de/blog/his-exchange-process-requirements-all-you-ever-wanted-know}{The HIS Exchange Process for Requirements–all you ever wanted to know}.
\end{info}

The ReqIF team expects that making the Requirements Interchange Format an OMG standard increases the number of interoperable exchange tool implementations on the market, fosters the trust of companies exchanging requirement information in the exchange format and provides safety of investments to tool vendors.

% -----------------------------------------------------------------------------------
\subsection{Previous Versions of ReqIF}
\index{RIF}
\label{sec:RIF}
% -----------------------------------------------------------------------------------

This document is submitted as RFC of the Requirements Interchange Format (ReqIF) to the OMG.  Before the submission, the Requirements Interchange Format has been a specification proposed by the HIS and in its latest version, a recommendation of ProSTEP iViP.  For these versions, the abbreviation ``RIF'' has been applied.  The HIS released the Requirements Interchange Format as RIF 1.0, RIF1.0a, RIF 1.1; RIF1.1a and the ProSTEP iViP released the recommendation RIF 1.2.

As the acronym RIF has an ambiguous meaning within the OMG, the acronym ReqIF has been introduced to separate it from the W3C`s Rule Interchange Format.  ReqIF 1.0 is the direct successor of the ProSTEP iViP recommendation RIF 1.2.

\begin{warning}
The ProR GUI does currently not support RIF.
\end{warning}

% -----------------------------------------------------------------------------------
\subsection{Internal Attributes}
\label{sec:reqif_internal_attributes}
\index{internal attributes}
\index{attributes!internal}
% -----------------------------------------------------------------------------------

ReqIF allows users to define the attributes that SpecObjects may carry.  In addition to these, there are a number of internal attributes that are defined by the ReqIF standard.  Examples include an internal ID, or the last change timestamp.

These internal attributes are rarely of interest to users who just want to work with requirements.  However, they may be of interest to tool experts, or may be inspected for troubleshooting.

Internal attributes can be accessed from the Properties View, using the \menu{All Attributes} tab.

% ===================================================================================
\section{Terminology}
\label{sec:terminology}
\index{Terminology}
% ===================================================================================

Working with \pror{} can be confusing, as it uses the terminology from ReqIF.  For instance, ReqIF uses \textit{SpecObject}s, rather than \textit{requirements}.  In the following, we define the more important terms.  More terms are defined throughout the document.  You can use the index to find the definition of terms.

\begin{info}
This book uses ReqIF terminology throughout.  Please refer to this chapter to understand the meaning of these terms.
\end{info}

