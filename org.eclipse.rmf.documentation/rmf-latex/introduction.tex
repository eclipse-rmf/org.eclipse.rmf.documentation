\pror{} is the user interface of the Eclipse Requirements Modeling Framework.  This handbook strives to be a complete reference for \pror{}.  Included is a tutorial making it as easy as possible for new users to get started with requirements engineering.

\section{Conventions}

Throughout this book you'll see the following conventions:

% Formatted like remark
\begin{info}
This is where you'll find useful information or tips.
\end{info}

\begin{warning}
This will mark things to be aware of.
\end{warning}

\begin{example}
Examples will be marked as such.
\end{example}


When referring to \menu{menus} or \menu{user interface elements}, they are marked as shown here.
%\begin{definition}[Definition name]
%This is a definition
%\end{definition}


\section{Contributing}

Documentation is one of those things that gets easily neglected in open source projects.  It is also one of the easiest for outsiders to contribute to.  The documentation is managed as \LaTeX, which may scare some people.  But no worries, those who don't want to learn \LaTeX don't have to.

There are broadly two ways for contributing to the documentation:

\begin{description}
  \item[File a bug.]  Visit the \href{https://bugs.eclipse.org/bugs/enter_bug.cgi?assigned_to=&blocked=&bug_severity=normal&bug_status=NEW&comment=&contenttypeentry=&contenttypemethod=autodetect&data=&dependson=&description=&flag_type-1=X&flag_type-11=X&flag_type-12=X&flag_type-2=X&flag_type-4=X&flag_type-6=X&flag_type-7=X&flag_type-8=X&form_name=enter_bug&keywords=&&op_sys=All&product=MDT.RMF&qa_contact=&rep_platform=All&short_desc=&version=unspecified}{RMF Bug Tracker}.  You can just point out a problem or request for improvement.  You can also provide some text to be added to the documentation (unformatted).  If you do, however, then you need to sign a Committer License Agreement (CLA)
  \item[Submit improved \LaTeX via Gerrit.]  If you are technically inclined (meaning that you know what \LaTeX and git are, and how to use them), then you can contribute via the Gerrit code review system, as described \href{https://wiki.eclipse.org/Gerrit}{at eclipse.org}.
\end{description}

\subsection{Gerrit for Contributions}

TODO - when done, update the parent section as well.

\section{Acknowledgements}

Many parties were involved in the creation of RMF.  We would like to thank the core team that made it possible.

\begin{figure}[H]
  \centering
  \includegraphics[width=\textwidth]{../rmf-images/2012_03_sprint_team.jpg}
  \caption{The RMF team during a Sprint in April 2012 in Düsseldorf, Germany (left to right) Lukas Ladenberger, Mark Brörkens, Ingo Weigelt, Said Salem, Michael Jastram}
  \label{fig:intro_core_team}
\end{figure}


The roots of this project were created by Andreas Graf, Michael Jastram and Nirmal Sasidharan, who joined together individual projects to create RMF.  Their efforts were financed by the research projects itea Verde and FP7 Deploy.  RMF was assembled at the Eclipse Foundation, where it has been active ever since.  Figure~\ref{fig:intro_core_team} shows four of the five RMF Committers at a joint coding session (missing is Andreas Graf).

\section{License}
\label{sec:license}

This work, or parts thereof, are licensed under the \href{https://www.eclipse.org/legal/epl-v10.html}{Eclipse Public License Version 1.0} as part of the Eclipse Requirements Modeling Framework (RMF) project at \href{https://www.eclipse.org/rmf}{eclipse.org/rmf}.


