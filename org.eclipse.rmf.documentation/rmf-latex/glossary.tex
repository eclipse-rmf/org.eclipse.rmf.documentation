%*******************************************************************************
% Copyright (c) 2014 Formal Mind GmbH and others
% All rights reserved. This program and the accompanying materials
% are made available under the terms of the Eclipse Public License v1.0
% which accompanies this distribution, and is available at
% http://www.eclipse.org/legal/epl-v10.html
% 
% Contributors:
%     Michael Jastram - initial Copy
%     Maha Jastram - susequent improvements
%******************************************************************************/

% ===================================================================================
\section{ReqIF}
% ===================================================================================

ReqIF stands for Requirements Interchange Format.  It is an exchange format for requirements and a data model.  ReqIF is the successor of \href{RMF/RIF}{RIF} and an OMG standard.

RMF supports ReqIF 1.0.1 (\href{http://www.omg.org/spec/ReqIF/}{official OMG standard}).

The \href{RMF/ProR}{ProR} GUI does support ReqIF.

% -----------------------------------------------------------------------------------
\subsection{History}
% -----------------------------------------------------------------------------------

For technical and organizational reasons, two companies in the manufacturing industry are rarely able to work on the same requirements repository and sometimes do not work with the same requirements authoring tools.  A generic, non-proprietary format for requirements information is required to cross the chasm and to satisfy the urgent industry need for exchanging requirement information between different companies without losing the advantage of requirements management at the organizations' borders.

The Requirements Interchange Format (ReqIF) described in this RFC defines such a tool-independent exchange format.  Requirement information is exchanged by transferring XML documents that comply to the ReqIF format.

In 2004, the HIS (Hersteller Initiative Software), a panel of Germany's automotive manufacturers (Daimler, VW, Porsche, Audi and BMW Group) developed the idea of creating the ``Requirements Interchange Format''.  In 2005, the first version of that format was presented at the REConf®, a conference about requirements engineering and management, in Munich.  In 2008, the HIS Steering Committee decided that the internationalization and maintenance of the Requirements Interchange Format should be proceeded with the ProSTEP iViP Association.  A project was set up and a team was built that includes members of the ProSTEP iViP Association, representatives of manufacturing companies (Audi, BMW  Group, Daimler, VW, Bosch and Continental), tool vendors (Atego, IBM, MKS) and development partners (HOOD GmbH, PROSTEP AG).

The ReqIF team expects that making the Requirements Interchange Format an OMG standard increases the number of interoperable exchange tool implementations on the market, fosters the trust of companies exchanging requirement information in the exchange format and provides safety of investments to tool vendors.

% -----------------------------------------------------------------------------------
\subsection{Previous Versions of ReqIF}
% -----------------------------------------------------------------------------------

This document is submitted as RFC of the Requirements Interchange Format (ReqIF) to the OMG.  Before the submission, the Requirements Interchange Format has been a specification proposed by the HIS and in its latest version, a recommendation of ProSTEP iViP.  For these versions, the abbreviation ``RIF'' has been applied.  The HIS released the Requirements Interchange Format as RIF 1.0, RIF1.0a, RIF 1.1; RIF1.1a and the ProSTEP iViP released the recommendation RIF 1.2.

As the acronym RIF has an ambiguous meaning within the OMG, the acronym ReqIF has been introduced to separate it from the W3C`s Rule Interchange Format.  ReqIF 1.0 is the direct successor of the ProSTEP iViP recommendation RIF 1.2.

% -----------------------------------------------------------------------------------
\section{RIF}
% -----------------------------------------------------------------------------------

RIF stands for Requirements Interchange Format.  It is an exchange format for requirements and a data model.  RIF is the predecessor of \href{RMF/ReqIF}{ReqIF}.

RMF supports the following versions of RIF:

\begin{itemize}
\item RIF 1.1a
\item RIF 1.2
\end{itemize}

The ProR GUI does currently not support RIF.

% ===================================================================================
\section{Eclipse}
% ===================================================================================

% -----------------------------------------------------------------------------------
\subsection{Committer License Agreement (CLA)}\index{Committer License Agreement}
% -----------------------------------------------------------------------------------

The Committer License Agreement (CLA) needs to be signed by contributors to Eclipse projects.  It essentially states that you hold all rights to your contribution and that you allow Eclipse to use them under the Eclipse Public License.


