%*******************************************************************************
% Copyright (c) 2014 Formal Mind GmbH and others
% All rights reserved. This program and the accompanying materials
% are made available under the terms of the Eclipse Public License v1.0
% which accompanies this distribution, and is available at
% http://www.eclipse.org/legal/epl-v10.html
% 
% Contributors:
%     Michael Jastram - initial Copy
%******************************************************************************/

The importance of requirements has been recognized for a long time.  And with the advent of computer-aided engineering tools, a number of proprietary solutions have popped up all over the place.  While this has helped organizations to manage their requirements more efficiently, interoperability became a major issue.

The development of the ReqIF standard for requirements exchange finally provided a standard, feature-rich way of accessing requirements data.  Eclipse was the obvious choice for a reference implementation of this open standard.  The result is the Eclipse Requirements Modeling Framework, a complete, open source, user-friendly implementation of ReqIF.

This handbook is a comprehensive documentation of the \pror{} tool, which is based on Eclipse RMF.  All answers with respect to tool use should be answered here.  Furthermore, it contains a small tutorial (Chapter~\ref{sec:tutorial}) to get you started quickly.

Keep in mind that tools are meant to support processes, not the other way around.  \pror{} is a flexible tool, and it can be tailored to support your processes.  But development processes are explicitly outside the scope of this handbook.

\begin{info}
If you are interested in adopting a lightweight development process that can be used with \pror{}, visit our initiative \href{http://re-teaching.org}{re-teaching.org}.
\end{info}

\section{RMF, ProR, Essentials and formalmind Studio}
\index{formalmind Studio}
\index{ProR Essentials}
\index{Essentials}

There are a few derivatives of the RMF project that may be confusing.  The following will
help you to understand the ecosystem and how the pieces fit together:

\begin{description}
\item[RMF.] The Requirements Modeling Framework (RMF) is an open source project that is managed by the Eclipse Foundation.  It consists of software code, documentation, mailing lists, online forum, etc.
\item[ProR.] ProR is the name of the user interface that allows users to work with ReqIF-based requirements.  ProR is typically installed into existing Eclipse installations.  There used to be a stand-alone build of ProR, which has been discontinued.
\item[RMF Core.] While ProR is the frontend of RMF, the Core is the back end.  This distinction is primarily intended for developers.
\item[ProR Essentials.] The company Formal Mind created this collection of add-ons that make ProR much more usable.  For instance, Essentials allows for the editing and rendering of formatted text.
\item[formalmind Studio.] As there is no stand-alone version of ProR available, Formal Mind created one, which comes with Essentials preinstalled.  For users who ``just want to edit requirements'', this is the most convenient way of getting started.
\end{description}

\section{Compatibility}

\pror{} and RMF require at least Java 6 and Eclipse 3.8.  Some features of ProR Essentials require Java 7 from Oracle (not openJDK).

formalmind Studio is based on Eclipse Luna (4.4).


